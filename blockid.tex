\documentclass{IEEEconf}
\usepackage{amsmath, amssymb}

\newtheorem{theorem}{Theorem}
\title{AcBF: A Revokable Blockchain-based Identity Management Scheme for Extremely Low-Latency Authentication Scenarios}
\author{Jianan Hong, Jia Cheng, Chuan Zhang, Tong Wu, Yue Wu}
\begin{document}
\maketitle

\section{Introduction}
Blockchain is an attractive technology for distributed identity management due to its security features, such as append-only and non-tempering storage, decentralized consensus, and reliable trust estabilishment. Compared with traditional identity system, such as PKI (public key infrastructure), the characture of blockchain-based method suits better in current and future network systems, where

\section{Construction}
\subsection{Revocation-Aware Block Format}
The block structure in this paper is desgined to make lightweight node aware of critical ledger status, e.g., user revocation. 

Block header \dots

[Two flag bits, consensus field is modeled as a PoW method]

Transactions \dots

[transactions are structured as normal MHT if no revocation happens. When any flag is ..., the difference is introduced]


\subsection{System Procedures}
\subsubsection{Initialization}
xxx

\subsubsection{Certificate Register}
xxx



\subsubsection{Revocation}

\subsection{Parameter Determine for Bloom Filter}

The length of Bloom Filter in this paper can be dramatically decreased compared with other schemes for certificate revocation. This section gives a theoretical derivation for the relevant parameter.

Let $n, \delta$ be the possible maximal number of registered and revoked users, respectively. The remaining factor is system effect of user revocation: when revocation is executed in one consensus round, the expected affected users should be limited to $\theta$, which can be measured as the member amount of Accumulator $\Delta$. The measurement is feasible as:
\begin{itemize}
    \item Larger member amount in $\Delta$ causes larger probability that a user to be revoked has already in $\Delta$ and more other members to update their auxiliary information as \eqref{xxx}.
    \item The probability that innocent users applies for accumulator member affects the increase rate of $\Delta$'s member, and this is also an effect to minimize.
\end{itemize}

Assume $m$ is the calculated hash length with $k$ hash functions in the Bloom filter. With $\delta$ users already be revoked, the probability can be measured as follows, that an legal user is erroneously claimed as a revoked one: 
\begin{align} 
    \Pr(1) \simeq (1 - e^{-k\delta/m})^k 
 \end{align}

The parameter $k$ can be individually optimized to minimize $\Pr(1)$ as $ k = \frac{m}{\delta}\ln 2$. Then, for $n$ valid users, the expected ... (limited to $\theta$) is 
\begin{align}\label{eq:bloomfilter}
    \theta = n \cdot (1 - e^{-k\delta/m})^{k} = 0.5^{\frac{m}{\delta}\ln 2}\cdot n
\end{align}
From \eqref{eq:bloomfilter}, the parameters of Bloom filter can be determined as 
\begin{align}
m = & \frac{\ln \frac{n}{\theta} \cdot \delta}{(\ln 2)^2} \simeq 2.08\delta \cdot\ln \frac{n}{\theta}\\
k = & \lfloor \ln \frac{n}{\theta} / \ln 2 \rceil \simeq \lfloor 1.44 \ln \frac{n}{\theta} \rceil
\end{align}

\section{Security Analysis of AcBF}
\begin{theorem}
	The AcBF identity management achieves sound and efficient user authentication for the lightweight-node verifier.
\end{theorem}
\section{Related Works}
\end{document}